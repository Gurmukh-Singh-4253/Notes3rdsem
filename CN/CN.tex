\documentclass[11pt,letterpaper]{article}

\usepackage{graphicx}
\usepackage{fancybox}
\usepackage[utf8]{inputenc}
\usepackage{epsfig,graphicx}
\usepackage{multicol,pst-plot}
\usepackage{pstricks}
\usepackage{amsmath}
\usepackage{amsfonts}
\usepackage{amssymb}
\usepackage{eucal}
\usepackage{upgreek}
\usepackage[left=2cm,right=2cm,top=2cm,bottom=1cm]{geometry}
\usepackage{tcolorbox}
\usepackage{import}
\pagestyle{empty}
\DeclareMathOperator{\tr}{Tr}
\renewcommand{\sp}[1]{$${\begin{split}#1\end{split}}$$}

\usepackage{lipsum}
\usepackage{mdframed}
\usepackage{listings}
\usepackage{color}

% Margins
% \topmargin=-0.45in
\evensidemargin=0in
\oddsidemargin=0in
\textwidth=6.5in
\textheight=9.0in
\headsep=0.25in

% \title{ Chemistry Notes}
% \author{ Gurmukh Singh }
% \date{\today}

 % The problem environment introduced.                                     
\newenvironment{problem}[2][Problem]                                  
        {\begin{tcolorbox}[colback=white,colframe=gray!50,title=#1 #2]}
        {\end{tcolorbox}}
        % {\begin{mdframed}[backgroundcolor=gray!20] \textbf{#1 #2} \\}
        % {\end{mdframed}}
% Define solution environment
\newenvironment{solution}                      
        {\begin{mdframed}\textit{Solution:} \\}
        {\end{mdframed}}
% Define an environments for proofs
\newenvironment{myproof} 
        {\textit{Proof:}}                                   
        {\begin{flushright} Q.E.D. \end{flushright}}
% Define a theorem environment and a notation one too
\newenvironment{mytheorem}                    
        {\begin{mdframed}\textbf{Theorem:} \\}
        {\end{mdframed}}
\newenvironment{notation}                      
        {\begin{mdframed}\textit{Notation:} \\}
        {\end{mdframed}}
% A new example wouldnt so any harm either...  
\newenvironment{example}                             
        {\textit{Example:}\\}
	{}
%I sholud be ashamed to forget the definition environment
\newenvironment{definition}
	{\begin{mdframed}$\underline{\textit{Def}^\textit{n}:} $\\}
	{\end{mdframed}}
%Corollary envvvvvvvvv
\newenvironment{corollary}
	{\textbf{Corrolary:}\\}

\pagestyle{empty}

\begin{document}

\begin{center}
  \Huge{Computer Networks Notes}\\
  \vspace{0.25cm}
  \small{Gurmukh Singh}
\end{center}

\vspace{-1.75cm}

\begin{flushright}
  Instructor: \\ Mrs. Shubhani Agarwal
\end{flushright}

\vspace{-1.3cm}

\begin{flushleft}
  B.Tech. CSE
\end{flushleft}

\rule{15.5cm}{0.1mm}%{\linewidth}{0.1mm}

% Optional TOC
\tableofcontents
\pagebreak

%--Paper--

\section{Computer Networks}

\subsection{Data Communication:}

the exchange of information between two devices via some form of transmission medium, such as wired cable. The 4 fundamental characteristics of this are:
\begin{itemize}
  \item Delivery
  \item Accuracy
  \item Time
  \item Jitter (variation in packet delivery time)
\end{itemize}

\subsection{Components of data communication:}
There are 5 components in data communication: 
\begin{itemize}
  \item Sender
  \item Reciever
  \item Message (the data to be transferred)
  \item Transmission medium
  \item Protocol(the rules and regulations to be send)
\end{itemize}

\subsection{Data representation:}
how we can represent data. It represents in the form of text, numbers, image, audio and video. 

\subsection{Data flow:}
There are 3 types of data flow: 
\begin{itemize}
\item Simplex
 	One way: sender stays sender and reciever stays reciever. 
	Unidirectional. 
\item Half-Duplex
	The direction can be changed but at a time there can only be one directional travel.
	Each station can both transmit and receive but not at the  same time. 
\item Duplex
	Simultaneous Data transfer over both nodes. 
	Both stations can transmit and receive simultaneously. 
\end{itemize}

\subsection{Networks:}
Network is a set of devices connected by communication links. A network must be able to meet a certain number of criteria. These are:
\begin{itemize}
  \item Performance (Throughput and latency)
  \item Reliability (low amount of downtime)
  \item Flexibility (Scalability)
  \item Security    (There should be no data manip)
\end{itemize}

\subsection{OSI model}
\begin{enumerate}
  \item Physical layer\\
    It is either Analog or Digital.\\
    It is a kind of continuous wave form that changes over time. 
    What matters in this is:
    For Analog:
    \begin{itemize}
      \item Amplitude
      \item Frequency
      \item Phase
    \end{itemize}
    For Digital:
    \begin{itemize}
      \item Bitrate
      \item Bit interval
    \end{itemize}
\end{enumerate}

\subsubsection{Transmissin impairment}
The quality of signal deteriorates during transmission. it may have three causes:
\begin{itemize}
  \item Attenuation
  \item Distortion
  \item Noise
\end{itemize}

\begin{definition}
   Attenuation is the loss of energy over the transmitted distance. It in measured in decibels (dB)
\end{definition}
\subsection{Layers of OSI model}subsection

\subsection{CRC: Cyclic Redundancy Check}
\begin{enumerate}
  \item This is used for error detection methods. item
  \item It is a very powerful method and widely used in real time enviroment. 
  \item It can detect all odd errors. 
  \item It can detect single and double bit errors. 
  \item It can detect burst error of length equals to polynomial degree. 
\end{enumerate}

Efficiency = Channel utilization for sending the message = $\frac{number of message bits}{total number of bits}$

\section{Forwarding and Routing Protocols}
Forwarding means sending packets in a "right" manner. 
\begin{enumerate}
  \item Cost effective
  \item Time efficient
  \item No congestion
\end{enumerate}
Routing protocols are of 3 types: 
\begin{enumerate}
  \item Unicast
  \item Multicast
  \item Broadcast
\end{enumerate}

\subsection{Unicast}
Unicast is of further two types :
\begin{enumerate}
  \item Intra-Domain
    \begin{enumerate}
      \item Distance version- RIP (Routing Information Protocol)
      \item Link State version - (OSPFC - Open Source Path First) (Uses Djikstras)
    \end{enumerate}
  \item Inter-Domain
    \begin{enumerate}
      \item Path vector (BGP)
    \end{enumerate}
\end{enumerate}

\end{document}
