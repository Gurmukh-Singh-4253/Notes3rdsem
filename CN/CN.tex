Computer Networks

Data Communication:
the exchange of information between two devices via some form of transmission medium, such as wired cable. The 4 fundamental characteristics of this are:
- Delivery
- Accuracy
- Time
- Jitter (variation in packet delivery time)

Components of data communication:
There are 5 components in data communication: 
- Sender
- Reciever
- Message (the data to be transferred)
- Transmission medium
- Protocol(the rules and regulations to be send)

Data representation:
how we can represent data. It represents in the form of text, numbers, image, audio and video. 

Data flow:
There are 3 types of data flow: 
- Simplex
 	One way: sender stays sender and reciever stays reciever. 
	Unidirectional. 
- Half-Duplex
	The direction can be changed but at a time there can only be one directional travel.
	Each station can both transmit and receive but not at the  same time. 
- Duplex
	Simultaneous Data transfer over both nodes. 
	Both stations can transmit and receive simultaneously. 

Networks:
Network is a set of devices connected by communication links. A network must be able to meet a certain number of criteria. These are:
- Performance (Throughput and latency)
- Reliability (low amount of downtime)
- Flexibility (Scalability)
- Security    (There should be no data manip)
