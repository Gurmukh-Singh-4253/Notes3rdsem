\documentclass[11pt,letterpaper]{article}

\usepackage{graphicx}
\usepackage{fancybox}
\usepackage[utf8]{inputenc}
\usepackage{epsfig,graphicx}
\usepackage{multicol,pst-plot}
\usepackage{pstricks}
\usepackage{amsmath}
\usepackage{amsfonts}
\usepackage{amssymb}
\usepackage{eucal}
\usepackage{upgreek}
\usepackage[left=2cm,right=2cm,top=2cm,bottom=1cm]{geometry}
\usepackage{tcolorbox}
\usepackage{import}
\pagestyle{empty}
\DeclareMathOperator{\tr}{Tr}
\renewcommand{\sp}[1]{$${\begin{split}#1\end{split}}$$}

\usepackage{lipsum}
\usepackage{mdframed}
\usepackage{listings}
\usepackage{color}

% Margins
% \topmargin=-0.45in
\evensidemargin=0in
\oddsidemargin=0in
\textwidth=6.5in
\textheight=9.0in
\headsep=0.25in

% \title{ Chemistry Notes}
% \author{ Gurmukh Singh }
% \date{\today}

 % The problem environment introduced.                                     
\newenvironment{problem}[2][Problem]                                  
        {\begin{tcolorbox}[colback=white,colframe=gray!50,title=#1 #2]}
        {\end{tcolorbox}}
        % {\begin{mdframed}[backgroundcolor=gray!20] \textbf{#1 #2} \\}
        % {\end{mdframed}}
% Define solution environment
\newenvironment{solution}                      
        {\begin{mdframed}\textit{Solution:} \\}
        {\end{mdframed}}
% Define an environments for proofs
\newenvironment{myproof} 
        {\textit{Proof:}}                                   
        {\begin{flushright} Q.E.D. \end{flushright}}
% Define a theorem environment and a notation one too
\newenvironment{mytheorem}                    
        {\begin{mdframed}\textbf{Theorem:} \\}
        {\end{mdframed}}
\newenvironment{notation}                      
        {\begin{mdframed}\textit{Notation:} \\}
        {\end{mdframed}}
% A new example wouldnt so any harm either...  
\newenvironment{example}                             
        {\textit{Example:}\\}
	{}
%I sholud be ashamed to forget the definition environment
\newenvironment{definition}
	{\begin{mdframed}$\underline{\textit{Def}^\textit{n}:} $\\}
	{\end{mdframed}}
%Corollary envvvvvvvvv
\newenvironment{corollary}
	{\textbf{Corrolary:}\\}

\pagestyle{empty}

\begin{document}

\begin{center}
  \Huge{Statistics Notes}\\
  \vspace{0.25cm}
  \small{Gurmukh Singh}
\end{center}

\vspace{-1.75cm}

\begin{flushright}
  Instructor: \\ Mrs. Neha
\end{flushright}

\vspace{-1.3cm}

\begin{flushleft}
  B.Tech. CSE
\end{flushleft}

\rule{15.5cm}{0.1mm}%{\linewidth}{0.1mm}

% Optional TOC
\tableofcontents
\pagebreak

%--Paper--

\section{Measures of Central Tendency}
\begin{enumerate}
  \item Mean
  \item Median
  \item Mode
\end{enumerate}

\subsection{Mean}
It is the ratio of sum of all the observations to the total number of observations.
let $x_1, x_2, \dots, x_n$ be all the observations. then:
\[
  \overline{x} =\frac{\sum_{i=1}^n x_i}{n}
\]

\subsubsection{Properties of Mean}
\begin{itemize}
  \item The sum of deviation of observations from mean is always zero
  \item the sum of square of deviations of observations is minimum as compared to any other measure.
  \item suppose there are two sequences:
    \[
      \begin{tabular}{c|c|c}
        & Series 1 & Series 2 \\ 
        Number of observations & $n_1$ & $n_2$ \\ 
        mean of the observations & $\overline{x}_1 $& $\overline{x}_2$
      \end{tabular} 
    \]
    then
    \[
      \overline{x} = \frac{n_1x_1+n_2x_2}{n_1+n_2}
    \]
\end{itemize}

\begin{problem}1
  If there are 5 and 8 number of observations of 2 series with mean 15 and 18, find the combined mean
\end{problem}

\begin{solution}
  We can get the solution by taking the weighted mean of the two sequences. \\ 
  so the required mean is : 
  \[
    \frac{5 \times 15 + 8 \times 18}{5+8}
  \]
  \[
    = \frac{75 + 144}{13}
  \]
  \[
    = \frac{219}{13}
  \]
  \[
    = 16.846154
  \]
\end{solution}

\begin{problem}2
  $$
   \begin{tabular}{c c}
     Class & frequency \\ 
   0-10 & 3 \\ 
   10-20 & 5\\ 
   20-30 & 7\\ 
   30-40 & 4\\ 
   40-50 & 1\\ 
   \end{tabular}
  $$
\end{problem}

\begin{solution}
  change of origin:
  $$
   \begin{tabular}{c c c c c }
     Class & frequency & X & d=X-A & f$\cdot $d\\ 
     0-10 & 3 & 5 & -20 &  \\ 
     10-20 & 5 & 15 & -10  & \\ 
     20-30 & 7 & 25 & 0 & \\ 
     30-40 & 4 & 35 & 10 & \\ 
     40-50 & 1 & 45 & 20 & \\ 
   \end{tabular}
  $$
  \[
    \overline{x} = A + \frac{\sum fd}{n}
  \]

  change of scale
  $$
   \begin{tabular}{c c c c c }
     Class & frequency & X & d=X/n & f$\cdot $d\\ 
     0-10 & 3 & 1 & -20 &  \\ 
     10-20 & 5 & 3 & -10  & \\ 
     20-30 & 7 & 5 & 0 & \\ 
     30-40 & 4 & 7 & 10 & \\ 
     40-50 & 1 & 9 & 20 & \\ 
   \end{tabular}
  $$
  \[
    \overline{x} = A + \frac{\sum fd}{n}
  \]
\end{solution}

\subsection{Median}
Steps to find Median in case of Discrete and continuous data:

\begin{enumerate}
  \item Arrangement of data
  \item if $n$ is odd then the median is the $ \frac{n+1}{2}$th term
  \item if $n$ is even then the median is the mean of the $\frac{n}{2}$th term and $\frac{n}{2}+1$th term
\end{enumerate}
\begin{problem}3
   find the median for the data : 
   \begin{enumerate}
      \item 9,9,10,10,12,13,15
      \item 9,9,10,10,12,13,14,15
   \end{enumerate}
\end{problem}

\begin{solution}
  \begin{enumerate}
    \item 9,9,10,10,12,13,15 has 7 elements. Therefore our median will be the 4th term in the arranged order\\
      $ \therefore Median = 10 $
    \item 9,9,10,10,12,13,14,15 has 8 elements. Therefore our median will be the mean of the 4th and 5th terms. \\
      $\therefore Median = \frac{10+12}{2}= 11$
  \end{enumerate}
\end{solution}

\begin{problem} 4
  Finding the median of discrete data. 
  \begin{center}
    \begin{tabular}{c|c|c}
      X & f & cf(cumulative frequency)\\
      \hline
      1 & 5 & 5\\ 
      2 & 8 & 13\\ 
      3 & 9 & 22\\ 
      \textbf{4} & \textbf{12} & \textbf{34}\\ 
      5 & 6 & 40\\ 
      6 & 7 & 47\\ 
      7 & 4 & 51\\ 
      \hline
      Total & 51 \\
    \end{tabular}
  \end{center}

  find the value of $x$ which has cumulative frequency just greater than $\frac{n}{2}$
\end{problem}

In case of continuous data:
\[
  Median = l + \frac{\left( \frac{n}{2} - cf \right)h}{f}
\]
where cf is the cumulative frequency and f is the frequency of the chosen class, $h$ is the class size

\subsection{Mode}
The observation which occurs the most is called the mode of the data. \\
In more general terms, the most probable observation in a dataset is the mode of the data. 

\begin{problem}5
  Find mode for the following data:
  10,11,15,18,18,18,15,10,18,20
\end{problem}

\begin{problem}6
  Find the mean, median and mode for the following data
  \begin{center}
    \begin{tabular}{c|c}
      CI & f \\
      \hline
      0-10 & 3 \\
      10-20 & 5 \\ 
      20-30 & 7 \\ 
      30-40 & 2\\ 
      40-50 & 1 \\ 
      \hline
      Total & 51 \\
    \end{tabular}
  \end{center}
\end{problem}
\textbf{How to find the mode for continuous data}
\begin{enumerate}
  \item Find the modal class which is having the maximum frequency.
  \item based on that input the values into the following formulae:
    \[
      mode = l + h \left( \frac{f_1 - f_2}{2f_1-f_0-f_2} \right)
    \]
\end{enumerate}
\subsection{The interconnection between the measures of central tendency}
\[
  Mode = 3 Median - 2 Mean
\]

\subsection{Geometric and Harmonic mean}
\begin{definition}
   Geometric mean is defined as the $n$th root of the product of $n$ observations\\
   Mathematically:
   \[
     GM = \sqrt[n]{\prod_{i=0}^n x_i}
   \]
\end{definition}

\begin{problem}7
   Find the Geometric Mean for the values 2,4,8
\end{problem}

\begin{definition}
   Harmonic mean is defined as the reciprocal of arithemetic mean of the reciprocal of all the observations
   \[
      HM = \frac{n}{\sum_{i=1}^{n} \frac{1}{x_i}}
   \]
\end{definition}

\begin{mytheorem}
   The following inequality is always true:
   \[
      AM \geq GM \geq HM
   \]
\end{mytheorem}

\end{document}
